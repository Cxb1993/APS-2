
\documentclass[11pt,a4paper]{article}


\usepackage[bookmarks,%
            a4paper,%
            breaklinks,%
            backref=false,%
            dvips,ps2pdf,%
            pdfhighlight=/I,%
            pdffitwindow=true,%
            pdfstartview=Fit,%
            pdfcenterwindow=true,%
            linkbordercolor={1 0 1},%
            %colorlinks,%
            pdftitle=Essential LaTeX Templates,%
            pdfauthor=Palas Kumar Farsoiya]%
            {hyperref}


\usepackage{amsmath}

\usepackage{natbib}
\usepackage{booktabs}
\usepackage{graphicx}
\usepackage{float}
\graphicspath{{images/}}




\newcommand{\etas}{\ensuremath{\eta_{\mathrm{s}}}}
\newcommand{\Rey}{\ensuremath{\mathrm{Re}}}
\newcommand{\avg}[1]{\ensuremath{\overline{#1}}}
\newcommand{\tenpow}[1]{\ensuremath{\times 10^{#1}}}

\newcommand{\pder}[2]{\ensuremath{\frac{\partial#1}{\partial#2}}}


\newcommand{\Eqref}[1]{Equation~\eqref{#1}}
\newcommand{\Tabref}[1]{Table~\ref{#1}}
\newcommand{\Figref}[1]{Figure~\ref{#1}}
\newcommand{\Appref}[1]{Appendix~\ref{#1}}





\begin{document}
\title{A study of Wave formation in the gravity driven low Reynolds number flow of two liquid films down an inclined plane by KangPing Chen}
\author{Palas Kumar Farsoiya}

\date{\vspace{-5ex}}

\maketitle


\begin{abstract}
Instability due to viscous stratification has been studied since the pioneering work by \cite{Yih1967}. \cite{Chen1993} work orginates from the results of 
 \cite{HB1987}, in which they investigated a problem of viscosity stratified two layer flow for low Reynolds number provided the layer 
adjacent to the wall is very thin compared to the upper layer. They found the flow is stable for large wavelength perturbations if 
the less viscous layer is adjacent to the wall(called it thin layer or lubrication effect). \cite{LL1989} took this problem and modified the
upper layer very thin compared to the lower layer which is adjacent to the wall and ignored the inertial effects. Now they found the flow is always unstable 
if the lower layer is less viscous. They suggested this is due to the presence of free surface above the upper layer and called it antilubrication effect.
\cite{Chen1993} examined the LL results in the limit to infinite upper layer thickness, and found the flow is neutrally stable which contrasts the HB result.
Chen intuited that might be the inertial effects that LL ignored will stablize the HB configuration. Chen extented the LL work to the case of finite thicker 
upper layer and then introduced the inertial effects in it. Chen concluded that increasing the thickness of upper layer reduces the antilubrication effect, and in
the limit of infinite upper layer thickness it was in concordance with HB's result. It was also found that if the less viscous layer is adjacent to the wall,
stability for surface and interface modes cannot be achieved simultaneously. However, for the opposite configuration (more viscous layer adjacent to the wall)
the flow can become linearly stable under certain conditions.
\end{abstract}
\pagebreak
\bibliographystyle{chemthes}
\bibliography{mylit}


\end{document}

















\documentclass[a4paper,10pt]{article}
\usepackage[utf8]{inputenc}

%opening
\title{Wave formation in the gravity driven low Reynolds number flow of two liquid films down an inclined plane}
\author{Palas Kumar Farsoiya}

\begin{document}

\maketitle

\begin{abstract}
Instability due to viscous stratification has been studied from many years 
\end{abstract}


\end{document}
